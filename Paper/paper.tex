\documentclass[12pt]{report}

\usepackage{amsmath}
\usepackage{amsfonts}
\usepackage{amssymb}
\usepackage{amsthm}
\usepackage{ulem} % underline
\usepackage{float} % H float specifier
\usepackage{hyperref} % hyperlinks
% \usepackage{hyphenat}
\usepackage{bookmark}
\usepackage{systeme} % systems of equation
\usepackage[utf8]{inputenc} % necessary for proper unicode support
\usepackage{parcolumns} % multiple column parallel typesetting
\usepackage[english]{babel} % comment this for Angelsächsisch
\usepackage{graphicx} % including graphics
\usepackage{sourcecodepro} % best font
\usepackage{listings} % source code listings
\usepackage{biblatex} % better citation
% \usepackage[a4paper, top=1cm, left=2cm, right=2cm, includeheadfoot]{geometry} % customize page geometry

% umlauts in lstlistings env
\lstset{
    literate={ö}{{\"o}}1
    {ä}{{\"a}}1
    {ü}{{\"u}}1
    {Ö}{{\"O}}1
    {Ä}{{\"A}}1
    {Ü}{{\"U}}1
}

\lstset{
    basicstyle=\footnotesize\ttfamily,
    breaklines=true
}

\lstset {
    numbers=left,
    stepnumber=1,
    firstnumber=1,
    numberfirstline=true
}

\providecommand{\lxor}{\veebar}
\renewcommand{\proofname}{Beweis.}
\newcommand{\sref}[1]{\textsuperscript{\ref{#1}}}
\newcommand{\F}[1]{\mathbb{F}_2^{#1}}

\addbibresource{refs.bib}

\title{
    \textbf{Efficient Implementation Strategies for Block Ciphers on ARMv8}\\
    {\footnotesize Bachelorarbeit}
}
\author{Bastian Engel}
\date{\today}

\begin{document}

\maketitle

\chapter*{Abstract}

Lorem ipsum dolor \cite{gift:2017} sit amet, consectetur adipisicing elit, sed do eiusmod tempor
incididunt ut labore et dolore magna aliqua. Ut enim ad minim veniam, quis
nostrud exercitation ullamco laboris nisi ut aliquip ex ea commodo consequat.
Duis aute irure dolor in reprehenderit in voluptate velit esse cillum dolore eu
fugiat nulla pariatur. Excepteur sint occaecat cupidatat non proident, sunt in
culpa qui officia deserunt mollit anim id est laborum.

\chapter*{Declaration}

I hereby declare that ...

\tableofcontents

\chapter{Introduction}
\section{Block ciphers}

Securing communication channels between different parties has been a long-term
subject of study for cryptographers and engineers which is essential to our
modern world to cope with ever-increasing amounts of devices producing and
sharing data. The main way to facilitate high-throughput, confidential
communications nowadays is through the use of symmetric cryptography in which
two parties share a common secret, called a key, which allows them to encrypt,
share and subsequently decrypt messages to achieve confidentiality against
third parties. Ciphers can be divided into two categories; block ciphers, which
always encrypt fixed-sized messages called blocks, and stream ciphers, which
continuously provide encryption for an arbitrarily long, constant stream of
data.

A block cipher can be defined as a bijection between the input block (the
message) and the output block (the ciphertext). For any block cipher with block
size $n$, we denote the key-dependent encryption and decryption functions as
$E_K,D_K:\F{n}\rightarrow \F{n}$. The simplest way to
characterize this bijection is through a lookup table which yields the highest
possible performance as each block can be encrypted by one simple lookup
depending on the key and the message. This is not practical though due to most
ciphers working with block and key sizes $n,|K|\geq 64$. For a block cipher
with $n=64,|K|=128$, a space of $2^{64}2^{128}64=2^{198}$ is necessary.
Considering modern consumer hard disks being able to store data in the order of
$2^{40}$, it is easy to see that a lookup table is wholly impractical. We
therefore describe block ciphers algorithmically which opens up possibilities
for different tradeoffs and security concerns.


\subsection{GIFT}

\texttt{GIFT}\cite{gift:2017}, first presented in the \textit{CHES 2017}
cryptographic hardware and embedded systems conference, is a lightweight block
cipher based on a previous design called \texttt|PRESENT|, developed in 2007. Its
goal is to offer maximum security while being extremely light on resources.
Modern battery-powered devices like RFID tags or low-latency operations like
on-the-fly disc encryption present strong hardware and power constraints. GIFT
aims to be a simple, low-energy cipher suited for these kinds of applications.

\texttt{GIFT} comes in two variants; \verb|GIFT-64| working with 64-bit blocks
and \texttt{GIFT-128} working with 128-bit blocks. In both cases, the key is 128
bits long. The design is a very simple, round-based substitution-permutation
network (SPN). One round consists in a sequential application of the confusion
layer by means of 4-bit S-boxes and subsequent diffusion through bit
permutation. After the bit permutation, a round key is added to the cipher
state and the single round is complete. \texttt{GIFT-64} uses 28 rounds while
\texttt{GIFT-128} uses 40 rounds.

\begin{figure}[h!]
    \centering
    \includegraphics[width=\textwidth]{Figures/GIFT-64.pdf}
    \caption{Two rounds of GIFT-64}
    \label{fig:gift64}
\end{figure}

\subsubsection{Substitution layer}

The input of \texttt{GIFT} is split into 4-bit nibbles which are then fed into
16 S-boxes for \texttt{GIFT-64} and 32 S-boxes for \texttt{GIFT-128}. The S-box
$S:\F{4}\rightarrow \F{4}$ is defined as follows:

\[
    \begin{array}{l|cccccccccccccccc}
        x & 0 & 1 & 2 & 3 & 4 & 5 & 6 & 7 & 8 & 9 & a & b & c & d & e & f \\
        \hline
        S(x) & 1 & a & 4 & c & 6 & f & 3 & 9 & 2 & d & b & 7 & 5 & 0 & 8 & e
    \end{array}
\]

\subsubsection{Permutation layer}

The permutation $P$ works on individual bits and maps bit $b_i$ to $b_{P(i)},
\forall i\in\{0,1,\dots,n-1\}$. The different permutations for \texttt{GIFT-64}
and \texttt{GIFT-128} can be expressed by:

\begin{align*}
    P_{64}(i)&=4\left\lfloor\frac{i}{16}\right\rfloor+16\left(\left(3\left\lfloor\frac{i\bmod 16}{4}\right\rfloor+(i\bmod 4)\right)\bmod 4\right)+(i\bmod 4) \\
    P_{128}(i)&=4\left\lfloor\frac{i}{16}\right\rfloor+32\left(\left(3\left\lfloor\frac{i\bmod 16}{4}\right\rfloor+(i\bmod 4)\right)\bmod 4\right)+(i\bmod 4) \\
\end{align*}

\subsubsection{Round key addition}

The last step of each round consists in XORing a round key $R_i$ to the cipher
state. The new cipher state $s_{i+1}$ after each full round is therefore given
by

\[
    s_{i+1}=P(S(s_i))\oplus R_i
\]

\subsubsection{Round key extraction and key schedule}

Round key extraction differs for \texttt{GIFT-64} and \texttt{GIFT-128}. Let
$K=k7||k6||\dots||k0$ denote the $128$-bit key state.

\paragraph{GIFT-64}. We extract two 16-bit words $U||V=k_1||k_0$ from the key
state. $u_i$ and $v_i$ are XORed to $r_{4i+1}$ and $r_{4i}$ of the round key
$R$ respectively.

\paragraph{GIFT-128}. We extract two 32-bit words $U||V=k_5||k4||k1||k_0$ from
the key state. $u_i$ and $v_i$ are XORed to $r_{4i+2}$ and $b_{4i+1}$ of the
round key $R$ respectively.

In both cases, we additionally XOR a round constant $C=c_5c_4c_3c_2c_1c_0$ to
bit positions $n-1,23,19,15,11,7,3$. The round constants are generated using a
6-bit affine linear-feedback shift register and have the following values:\\

\begin{tabular}{r|l}
    \textbf{Rounds} & \textbf{Constants} \\
    \hline
    \textbf{1 - 16} &  \small\texttt{01,03,07,0F,1F,3E,3D,3B,37,2F,1E,3C,39,33,27,0E} \\
    \textbf{17 - 32} & \small\texttt{1D,3A,35,2B,16,2C,18,30,21,02,05,0B,17,2E,1C,38} \\
    \textbf{33 - 48} & \small\texttt{31,23,06,0D,1B,36,2D,1A,34,29,12,24,08,11,22,04}
\end{tabular}\\

The key state is then updated by setting $k_1\leftarrow k_1\ggg 2$,
$k_0\leftarrow k_0\ggg 12$ and rotating the new state $32$ bits to the right:

\[
    k_7||k_6||\dots||k_1||k_0\leftarrow k_1\ggg 2||k_0\ggg 12||k_7||k_6||\dots||k_3||k_2
\]

\subsection{Camellia}

\section{The ARMv8 platform}

With small devices, embedded processors and ASICs becoming ever more ubiquitous
and essential in areas like medicine or automotive design, the need for ...

\chapter{Implementation strategies}

Due to the structural differences of SPN- and Feistel network-based ciphers, we
shall analyze these two separately.

\section{Strategies for SPN}

Three implementation strategies for substitution-permutation networks are
introduced by \cite{implx86:2014}:

\begin{itemize}
    \item Table-based implementations
    \item \texttt{vperm} implementations
    \item Bitslice implementations
\end{itemize}

\subsection{Table-based}

Table-driven programming is a simple way to increase performance of operations
by tabulating the results, therefore requiring only a single memory access to
acquire the result. This approach is obviously limited to manageable table
sizes, so while tabulating a function like the AES S-box
$S_{AES}:\F{8}\rightarrow \F{8}$ requires only $2^{11}$ space,
tabulating the \texttt{GIFT} permutation layer
$P_{GIFT}:\F{64}\rightarrow \F{64}$ would require
$2^{70}$ space, which is totally unfeasible.

A common approach is to tabulate the output of each S-box, including the
diffusion layer, and then XORing the results together. Let $n$ denote the
internal cipher state size and $s$ the size of a single S-box in bits. For each
S-box $S_i,i\in\{0,\dots,\frac{n}{s}\}$, we can construct a mapping
$T_i:\F{s}\rightarrow \F{n}$ representing substitution with subsequent
permutation of that single S-box. The cipher state before round key addition is
then given by $\bigoplus_{i=0}^{\frac{n}{s}-1}{T_i(m_i)}$ for each $s$-bit
message chunk $m_i$. This approach requires space of
$\frac{n}{s}|\F{s}|n=\frac{n^2 2^s}{s}$ bits, which, for \texttt{GIFT-64},
results in a manageable size of $\frac{64^2 2^4}{4}=2^{14}$ bits which equals
$16$ KiB.

\subsubsection{Constructing the tables}

For \texttt{GIFT-64}, table construction is relatively straightforward and can
be done as follows:

\begin{lstlisting}[frame=single, caption={Table construction algorithm}, escapechar=|]
    tables <- [][]
    for sbox_index from 0 to 15 do
        for sbox_input from 0 to 15 do|\label{lst:tablesbox}|
            output <- sbox(sbox_input)
            output <- permute(output << (4 * sbox_index))
            tables[sbox_index][sbox_input] <- output
\end{lstlisting}

Implementing this algorithm gives us the following table representing the first
and second S-box.

\[
    \begin{array}{l|l|l|c}\label{table:gift64subperm}
        x & T_0(x) & T_1(x) & \dots \\
        \hline
        0x0 & 0x1               & 0x1000000000000   & \dots \\
        0x1 & 0x8000000020000   & 0x800000002       & \dots \\
        0x2 & 0x400000000       & 0x40000           & \dots \\
        0x3 & 0x8000400000000   & 0x800040000       & \dots \\
        0x4 & 0x400020000       & 0x40002           & \dots \\
        0x5 & 0x8000400020001   & 0x1000800040002   & \dots \\
        0x6 & 0x20001           & 0x1000000000002   & \dots \\
        0x7 & 0x8000000000001   & 0x1000800000000   & \dots \\
        0x8 & 0x20000           & 0x2               & \dots \\
        0x9 & 0x8000400000001   & 0x1000800040000   & \dots \\
        0xa & 0x8000000020001   & 0x1000800000002   & \dots \\
        0xb & 0x400020001       & 0x1000000040002   & \dots \\
        0xc & 0x400000001       & 0x1000000040000   & \dots \\
        0xd & 0x0               & 0x0               & \dots \\
        0xe & 0x8000000000000   & 0x800000000       & \dots \\
        0xf & 0x8000400020000   & 0x800040002       & \dots
    \end{array}
\]

The tables for \texttt{GIFT-128} can be generated in a similar way by looping
through all 32 S-boxes instead of 16 on line \ref{lst:tablesbox}.

\subsection{Using \texttt{vperm}}

Nowadays, most instructions set architectures support single-instruction,
multiple-data processing. The idea of such an SIMD system is to work on
multiple data stored in vectors at once to speed up calculations. For A64, two
types of vector processing are available:

\begin{enumerate}
    \item Advanced SIMD, known as NEON
    \item Scalable Vector Extension (SVE)
\end{enumerate}

We will take a look at NEON as this is the type of vector processing supported
by the Cortex A-73 processor.

\subsubsection{ARM Neon}

The register file of the NEON unit is made up of 32 quad-word (128-bit)
registers \texttt{V[0-31]}, each extending the standard 64-bit floating-point
registers \mbox{\texttt{D[0-31]}}. These registers are divided into equally
sized lanes on which the vector instructions operate. Valid ways to interpret
for example the register \texttt{V0} are:

\begin{figure}[h!]
    \centering
    \includegraphics[width=\textwidth]{Figures/V_register.pdf}
    \caption{Divisions of the V register}
    \label{fig:gift64}
\end{figure}

NEON instructions interpret their operands' layouts (i.e. lane count and width)
through the use of suffixes such as \texttt{.4S} or \texttt{.8H}. For instance,
adding eight 16-bit halfwords from register \texttt{V1} and \texttt{V2}
together and storing the result in \texttt{V0} can be done as follows:

\begin{center}
    \texttt{ADD V0.8H, V1.8H, V2.8H}
\end{center}

\begin{figure}[h!]
    \centering
    \includegraphics[width=\textwidth]{Figures/vector_add.pdf}
    \caption{Addition of two vector registers}
    \label{fig:gift64}
\end{figure}

The plenitude of different processing instructions allow flexible ways to
further speed up algorithms having reached their optimizational limit on
non-SIMD platforms. \texttt{vperm}, a general term standing for \textit{vector
permute}, is a common instruction on SIMD machines. Called \texttt{TBL} on
NEON, it is used for parallel table lookups and arbitrary permutations. It takes
two inputs and performs a lanewise lookup:

\begin{enumerate}
    \item A register with lookup values
    \item Two or more registers containing data
\end{enumerate}

\subsubsection{S-box lookup}

This instruction can be used to implement S-box lookup of all 16 S-boxes in a
single instruction. We do this by packing our 64-bit cipher state
$s=s_{15}||s_{14}||\dots||s_0$ into a vector register $V_0$. Because we can
only operate on whole bytes, we put each 4-bit S-box into an 8-bit lane. We
then put the S-box itself into register $V_1$ which will be used as the data
register for the table lookup.

\subsection{Bitslicing}

\chapter{Implementation}

\chapter{Evaluation}

\section{Limitations}

\section{Benchmarks}

\chapter*{Acknowledgements}

I want to thank ...

\printbibliography

\end{document}
